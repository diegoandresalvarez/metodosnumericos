\documentclass[letterpaper,12pt]{exam}

\usepackage[english,spanish]{babel}
\usepackage[utf8x]{inputenc}
\usepackage[T1]{fontenc}
%\usepackage{fourier}
\usepackage{amsmath,amssymb,amsfonts,amsthm,bm}
\usepackage{mathtools}  
\mathtoolsset{showonlyrefs} 
\usepackage{graphicx}
\usepackage[colorinlistoftodos]{todonotes}
\usepackage[letterpaper,margin=2cm]{geometry}
\definecolor{OliveGreen}{rgb}{0.14,0.7,0.14} % 34-139-34
\definecolor{light-gray}{gray}{0.99}
\usepackage{listings}       % write source code in LaTeX
\usepackage{url}
\decimalpoint
\newcommand{\matlab}{$\text{MATLAB}^{\text{\textregistered}}$~}

\begin{document}

\begin{center}
\fbox{\fbox{\parbox{5.5in}{\centering
\textbf{Métodos Numéricos Aplicados a la Ingeniería Civil}.\\
Profesor: Felipe Uribe Castillo.\\
Taller III: Integración y Solución de ODEs.\\
Versión No. 2}}}
\end{center}

\vspace{1cm}
\noindent El plazo para la entrega del taller es hasta el \emph{Miércoles 4 de Mayo (6:00pm)}. Por cada hora de retraso en la entrega del trabajo se les descontará 0.1 unidades en la nota final. 

\begin{enumerate}

 \item La función de densidad de probabilidades de una variable aleatoria Gaussiana se define como,
 \begin{equation}
  f_{X}(x) = \frac{1}{\sigma\sqrt{2\pi}}\exp\left(-\frac{(x-\mu)^2}{2\sigma^2}\right)
 \end{equation}
 
 Calcule la probabilidad de que la variable aleatoria $X$ con media 5 y desviación estandar 8 se encuentre entre 4 y 15. Para ello, use $N = 3$ puntos de la cuadratura de Gauss-Legendre ($\boldsymbol{\xi} = [-\sqrt{3/5},~0,~\sqrt{3/5}]$). Use la ecuación vista en clase para calcular a mano los pesos de la cuadratura, deberá hallar los polinomios de Legendre usando la fórmula Rodrigues. Use comandos de integración de \matlab para validar el resultado (\texttt{integral} ó \texttt{quad}).

 \item Sobre un rascacielos el viento ejerce una fuerza distribuida, la cual se ha medido en diferentes alturas:
 \begin{table}[!hb]
  \centering
  \begin{tabular}{c|ccccccccc}
   $h$ [m]   & 0 & 30  & 60   & 90   & 120  & 150  & 180  & 210  & 240 \\
   \hline
   $w$ [N/m] & 0 & 340 & 1200 & 1600 & 2700 & 3100 & 3200 & 3500 & 3800
  \end{tabular}
 \end{table}
 
 Calcule la fuerza resultante de dicha carga distribuida ($F = \int_{a}^{b} w(x)\mathrm{d}x$) y su punto de acción (centroide del área bajo la curva: $\bar{x}=\int_{a}^{b} x w(x)\mathrm{d}x/\int_{a}^{b} w(x)\mathrm{d}x$). Utilice el método de Romberg con $k=3$ y $n=4$ a mano. Compare sus resultados con el programa visto en clase (para generar la ``function handle'' de estos puntos use el comando de \matlab \texttt{interp1}, ó bien modifique el programa para que acepte los puntos evaluados de la función, en lugar de la función).


\item La ODE que describe un oscilador no lineal de van der Pol está dada por:
 \begin{equation}
  x''(t) + \mu (x^2(t) - 1) x'(t) + x(t) = 0 \quad \text{con } x(t_0) = x'(t_0) = 0 \quad \text{y } \mu=1
 \end{equation}

 Programe el método de RK $3/8$ (use como base los programas hechos en clase), para resolver esta ecuación diferencial de segundo orden en el intervalo $0\leq t \leq 10$ usando un $h=0.25$ y $h=0.125$. Para ello habrá que plantear el espacio de estados. Compare sus resultados con la función de \matlab \texttt{ode45}. Mostrar los resultados en una sola gráfica para cada $h$. 
 
\end{enumerate}

\end{document}